\chapter{Stochastic Models of Packet Arrivals}
\label{models}

To simulate packet arrivals we must first define models of packet
arrivals.  Most of the early work in this area used the Poisson
process \cite{grimmett} as the basis of its models.  Many more complex
models have been used but many rely on the Poisson process somewhere
for the packet arrival distribution.  The Poisson process has the
advantage of being well understood and mathematically tractable.


The Poisson process is related to the exponential distribution (\S
\ref{append:expon}).  This distribution is one of the most widely
used in statistics and the basis for several others.  One feature of
the distribution is it has a finite mean and variance (\S
\ref{append:moments}).  The exponential distribution has the property
that the time spent waiting for an event (that will occur within a
time that is exponentially distributed) will not give you any
information about when it will occur.  This is known as the {\em
memoryless} property of the exponential (and its discrete counterpart
the geometric) distribution.

One important area of interest are processes which have inter-renewal
times with infinite moments.  We are mainly interested in those
distributions with infinite variance such as Cauchy,
$\mbox{t}_2\mbox{-distribution}$ and Pareto.

The research into infinite variance distribution renewal processes has
led to the idea of {\em fractal}, or {\em self similar} renewal
processes.  The basis of these names has to do with similar behaviour
at different time scales.  Poisson processes become smoother as you
observe their average behaviour over longer time intervals.  Pure
fractal processes do not smoothe so that they are invariant with
respect to the time period over which you observe them.

Such behaviour has serious consequences when applied to traffic models
because it indicates that you will always get bursts of traffic at
every time scale and you can never allocate enough queueing resources
to cope with every situation.

Much of the original work into fractals was done by Mandelbrot
\cite{IEEE:Mandelbrot} but it is only recently that his work has been
applied to communications traffic by researchers at Bellcore
\cite{Bell:1},\cite{Bell:2}, \cite{Bell:3}, \cite{Bell:4}.

The use of finite variance distributions was challenged in
\cite{Bell:1}.  In this thesis we examine the behaviour of Poisson
process based models and demonstrate that they are inadequate for
modelling network traffic observed on the University of Auckland
campus network.

The researchers at Bellcore produced traces for Ethernet LAN traffic,
ISDN packet traffic over public switched telephone systems and the
output of variable bit rate video encoders.  The earlier chapters of
this thesis attempted to duplicate their recording of Ethernet traffic
and produce comparable results as confirmation of their work.

The Bellcore work then went on to examine many of the self-similar
properties of the traffic and its relationship to theoretical work in
fractal behaviour.  In this thesis we simulate various statistical
models to see whether the observed behaviour can be reproduced.

\section{Poisson process}

\subsection{Definition}

A Poisson process is a counting process
\[ N = \{ N(t) : t \geq 0 \} \]
taking values from
\[ S = \{ 0,1,2,3, \ldots \} \]
conditioned on
\[N(0) = 0\]
and
\[ \mbox{if } s < t \mbox{ then } N(s) \leq N(t) \; \forall s,t \in {\mathbb R} \]
defined by
\[
\begin{array}{c}
P(N(t+h) = n + m | N(t) = n) = \\
P(N(t+h) - N(t) = m) = \\
\left\{
\begin{array}{rl}
1 - \lambda h + o(h) & m = 0 \\
\lambda h + o(h) & m = 1 \\
o(h) & m > 1 \\
\end{array}
\right. \\
\end{array}
\]

\subsection{Remarks}

The Poisson Process is one of the most common and simple stochastic
processes in statistics and operations research.  Poisson processes
occur in both time and space and are characterised by a single
parameter $\lambda$.  This parameter $\lambda$ is a measure of the
rate or intensity of the process.

Poisson processes in time occur along a real valued time line,
normally ${\mathbb R}^+$.  They create single events along the line as they
advance towards $+\infty$.

The above definition has the following consequences.

\begin{itemize}
\item	No event may occur at exactly time zero.

\item	The numbers of events in two non-overlapping time periods are
independent.  For example knowing the number of events in the period
$(0, t]$ will not give you any extra information about the number of
events in the period $(t, t+s]$, that is $N(t+s) - N(t) \sim N(s) \;
\forall s,t \in {\mathbb R}^+$.

\item	The number of events in the period $(0, t]$ is distributed
as a Poisson$(\lambda t)$ random variable, that is $P(N(t) = x) =
\frac{e^{-\lambda t}{\lambda t}^x}{x!}$, independently of all other
inter-event times.

\end{itemize}

It can easily be shown that these give us the following.

\begin{itemize}
\item	The time between any two successive events is distributed
$\mbox{Exponential}(\lambda)$, that is $P(T_i - T_{i-1} \geq t) =
e^{-\lambda t}$.

\end{itemize}

\section{Renewal processes}

A renewal process can be seen as a generalisation of a Poisson
process.  The distribution of the times between renewal events are
independent and identically distributed random variables.  For example
the Poisson process is a renewal process because the inter-arrival
times are independent and identically distributed (as exponentially
distributed random variables).  Renewal processes periodically
restart themselves at the renewal points, forgetting their previous
history.

The behaviour between renewal events may be complex.  Often the
renewal event is a marked event within a more complex process.  An
example is the simple random walk where the return to the origin or
return to zero can be used as the renewal event.  This is to maintain
consistency with the concept that when a renewal occurs it is as if
the process were starting from time zero.

\subsection{Simple renewal processes}

\subsection{Definition}

Let
\[ T = \{T(k) : k \geq 0 \; k \in {\mathbb N} \} \]
where $T(k)$ is the time of the $k^{th}$ event, conditioned on
\[ T(0) = 0 \]
Let
\[ \begin{array}{c}
R_k = T(k) - T(k-1) \\
R_k > 0 \; \;  k \geq 1 \\
\end{array}
\]
Then this is a renewal process if there exists a
distribution function $F_R(x)$ such that

\[ P(R_k \leq x) = P(R \leq x) = F_R(x) \; \forall x > 0 \mbox{ and } \forall k\geq 1 \]
Furthermore, the $R_k$ are mutually independent of one another.  That
is, inter-event times are independent and identically distributed.

\subsection{Remarks}

In simple renewal processes the process restarts every time an event
occurs.  This is the simplest type of renewal process and is very
similar to the Poisson process.  The main difference is that the
Poisson process relies on events having an exponentially distributed
inter-arrival time.  In a renewal process any probability distribution
may be used for the time until the next renewal.

Because the time until the next renewal has to be non negative
many common distributions such as the normal or Cauchy have to be
modified.  This can be done by folding the negative half of the
density onto the positive real line.  The simplest method is sampling
from the distribution and then taking the absolute value.

\subsection{Processes with internal structure}

Many processes are more complex then single events.  Between renewals
the process may produce other events.  A M/M/1 queue is a simple
queueing process where the time between arrivals is exponentially
distributed and the time taken for the customer at the head of the
queue to be served and leave is also exponentially distributed.  We
can make the renewal event the event of the system becoming empty.
The times when the system becomes empty form a renewal process with
the queueing process becoming hidden.

\section{Modulated processes}

Modulated processes are multi-state stochastic processes.  They
consist of individual states and a set of stochastic transitions which
govern the process' change from one state to another.

The process is in exactly one of the states at any given time.  While
it is in a state it acts in accordance with the behaviour that state
specifies.  The process may change state at any time and must enter
another state straight away.  The rules governing how long the process
stays in a state and which states can follow it may be state dependent
or be controlled by an underlying global process.

The difference between modulated processes and renewal process is that
each state in the modulated process has an associated renewal process
which produces events.  When a new state is entered a new renewal
process corresponding to that state is created rather than each
renewal process continuing to run but being hidden, producing events
which are then discarded.

For Markov modulated Poisson processes there is no difference because
of the memoryless property of the exponential distribution but for
more general processes this distinction is important.  This is due to
the {\em edge effects} created by the renewal process creation at
state changes.  If the lifetimes for the states are short causing
rapid state changing then these edge effects may seriously affect the
overall behaviour of the process.

\subsection{Markov modulated Poisson processes}

\subsubsection{Definition}

Let

\[ \mathcal{S} = \{s : s = 0,1,2,3,\ldots, N \} \]

be a set of states with associated arrival rates $\lambda_s$, where
$\lambda_s$ is the arrival rate of the underlying Poisson process
while in state $s$.  Let

\[ N = \{ N(t) : t \geq 0 \} \]

be the underlying Poisson process of rate $\lambda_{S(t)}$.  Let

\[
S = \{ S(t) : S(t) = s \in \mathcal{S} \; \; \forall t \geq 0 \; \; t \in {\mathbb R} \}
\]

be the state of the process at time $t$.  Let

\[
P(S(t+\delta) = k, \; 0 < \delta \leq h | S(t) = k) =
e^{- \mu_k h}
\]
where $\mu_k \in {\mathbb R}^+$ are state lifetime parameters, with
\[
\begin{array}{c}
P(S(t+h) = k | S(t) = j, h = \mbox{inf}\{S(t + \delta) \neq j \;,
\delta > 0\}) = M[j,k] \; \forall j,k \in {\mathcal S}, \\
M = M[m,n] \\
\end{array}
\]
is a stochastic $|\mathcal S| \times |\mathcal S|$ matrix such that
\[
\sum_n M[m,n] = 1 \; \; \forall m \in {\mathcal S}
\]

\subsubsection{Remarks}

Markov modulated Poisson processes are the simplest of the modulated
processes.  Each state defines the arrival rate of an underlying
Poisson process.

The amount of time the process stays in a state is exponentially
distributed with a stochastic transition matrix governing the next
state entered.

Markov modulated Poisson processes are often used to generate the
arrivals for standard queueing models.  This is so that more
variability is introduced into the model and to add extra realism
rather than having a constant arrival rate.

\subsection{Generalised modulated processes}
\label{models:genmodproc}

\subsubsection{Definition}

Let
\[ {\mathcal S} = \{s : s = 0,1,2,3,\ldots, N \} \]
be a set of state with associated distribution functions
\[
L_s : (0, +\infty) \rightarrow [0,1] \; s \in {\mathcal S}
\]
Let
\[
S = \{ S(t) : S(t) = s \in {\mathcal S} \; \; \forall t \geq 0 \; \; t \in {\mathbb R} \}
\]
be the state of the process at time $t$, $c(t)$ be the number of state
changes by time $t$, $\sigma \in {\mathcal S}$ be a starting state $S(0) =
\sigma$ and $\tau_n$ be the time of the $n^{th}$ change of state.
\[
\begin{array}{rcl}
\tau_0 & = & 0 \\
\tau_{n+1} & = & \mbox{inf}\{t : t > \tau_n, S(t) \neq S(\tau_n) \} \\
\\
c(t) = n & \mbox{iff} & \tau_n  \leq t \mbox{ and } \tau_{n+1} > t \\
\end{array}
\]
Let
\[ N_i = \{N_i(t) | N_i(0) = 0 \} \]
be a collection of simple renewal processes where $N_i(t) = k$ if $k$
events have occurred by time $(0,t]$ and $i \in {\mathcal S}$.  These are
template renewal processes.

Let
\[
P(\tau_{\alpha + 1} - \tau_\alpha > h) = 1 - L_{S(\tau_\alpha)}(h)
\]
with
\[
P(S(\tau_{\alpha + 1}) = k \;| \; S(\tau_\alpha) = j) = M[j,k] \; \forall j,k \in {\mathcal S}
\]
where
\[
M = M[m,n] \; \; 0 \leq m , n \leq |{\mathcal S}|
\]
is a stochastic matrix such that
\[
\sum_n M[m,n] = 1 \; \;  m \in {\mathcal S}.
\]
Let the current renewal process at time $t$ be
\[
Q_\alpha = \{Q_\alpha(t) \;| \; Q_\alpha(\tau_\alpha) = 0\} \mbox{ where } \tau_{\alpha} \leq t < \tau_{\alpha + 1}
\]
where
\[
P(Q_{\alpha}(t) = k) = P(N_{S(t)}(t - \tau_\alpha) = k)
\]
so that the total number of events by time $t$ is given by
\[
R(t) = \sum^{c(t)-1}_{i=0}{Q_i(\tau_{i+1})} + Q_{c(t)}(t)
\]
\subsubsection{Remarks}

A generalised modulated process consists of a set of states and a
stochastic transition matrix but unlike the Markov modulated Poisson
process it is not limited to only using the exponential distribution
as a lifetime distribution.

Each state has two distributions associated with it.  One is a renewal
process which generates events and the other is a lifetime
distribution.  When the process enters the state a lifetime is
generated from the lifetime distribution.  The process will remain in
that state for the period of the lifetime.

While in that state a renewal process will run generating events.  If
an event is generated which would have occured beyond the lifetime of
the state then it is ignored.  Any generalised renewal process is able
to run provided it is independent of all other states.

Once the lifetime of a state has expired then the process chooses a new
state to enter based on the probabilities in the transition matrix.
The transition matrix is constant and independent of any activity
within any of the states.

\section{Merged Processes}
\label{models:mergedprocs}

\subsection{Definition}

Let
\[ {\mathcal C} = \{c:c = 0,1,2,3,\ldots, N\} \]
be a set with associated general renewal processes
\[R_c(t), \; c \in {\mathcal C} \]

Let
\[ T_c = \left< T_{c_1}, T_{c_2}, \ldots \right> \]
 be the sequence of renewal times of the process $R_c$.

Let 
\[ T = \left<T_1, T_2, T_3, \ldots \right> = \bigcup^{\mathcal C}T_c \]
 be an ordered sequence such that 
\[ T_i \leq T_j \; \forall i,k \in {\mathbb N} \]
then T is a merged process.

\subsection{Remarks}

A merged process consists of multiple renewal or modulated processes
running concurrently with their resulting event traces merged
together.  In most models each process is identical and all must be
mutually independent.

The idea behind merged (or superimposed) processes is to capture
complex behaviour using simply renewal processes with few parameters.
This would give us a much simpler model without losing the complex
nature of the real behaviour.

\section{Slowly decaying variances}

\subsection{Processes with infinite moments}

There are several well known distributions with infinite moments.  The
t-distribution has infinite variance for $\nu = 2$ and infinite mean
and variance for $\nu = 1$ (this is also known as the Cauchy
distribution).  Another is the Pareto distribution.  This is a less
common distribution and is not often used in the area of stochastic
process modelling.

\subsubsection{Definition of the Pareto distribution}

\[
F_X(x) = P(X \leq x) = 1 - \frac{\beta^\alpha}{(\beta + x)^\alpha}
\]
and
\[
f_X(x) = \frac{\alpha \beta}{(\beta + x)^{\alpha + 1}}
\]
so that $\alpha = 2.5$ with give a distribution with finite mean and
variance, $\alpha = 1.5$ will give infinite variance and $\alpha =
0.5$ will have both infinite mean and variance.

\subsection{Aggregating stochastic processes}
\label{models:aggregation}

One of the methods for testing whether a sample has been generated
from a Poisson process is to see how the variance behaves under
averaged aggregation \cite{Bell:5}.

Let
\[
X = (X_1, X_2, X_3, \ldots)
\]
be a stationary stochastic process (for us $X_k$ will denote the
number of events per time unit).  For each $m = 1,2,3,\dots$ let
\[
X^{(m)} = (X^{(m)}_k : k = 1,2,3,\ldots) 
\]
denote a new (aggregated) time series obtained by averaging the original
series $X$ over non-overlapping blocks of size $m$.  That is for $m =
1,2,3,\ldots$, $X^{(m)}$ is given by
\begin{equation}
X^{(m)}_k = \frac{1}{m}(X_{km - m + 1} + \cdots + X_{km}) \; (k = 1,2,3,\ldots)
\label{models:eq1}
\end{equation}
Note that $X^{(m)}$ defines a new stationary process for each $m$.
 

Consider any collection of random variables which have the following
properties

\begin{itemize}

\item	All variables are independent and identically distributed.

\item	Their probability distribution have finite mean $\mu$ and
variance $\sigma^2 \in {\mathbb R}^+$.

\end{itemize}

Let
\[
X = (X_1,X_2,\dots,X_n)
\]
and
\[
{\bar X} = \sum^n_{i=1}\frac{X_i}{n} = \frac{1}{n}\sum^n_{i=1}X_i
\]
then
\[
\mu_{\bar X} = \mbox{E}({\bar X}) = \mbox{E}( \frac{1}{n}\sum^n_{i=1}X_i)
= \frac{1}{n}\sum^n_{i=1}\mbox{E}(X_i) = \mu
\]
and
\[
\sigma^2_{\bar X} = \mbox{Var}({\bar X}) = \mbox{Var}(\frac{1}{n}\sum^n_{i=1}X_i) = \frac{1}{n^2}\sum^n_{i=1}\mbox{Var}(X_i) = \frac{\sigma^2}{n}
\]

From above we can see that for aggregated renewal processes with finite
mean and variance the variance is of order $n^{-1}$ under
averaging.  The decreasing of variance is called {\em decaying
variance}.  So from equation~\ref{models:eq1} we obtain

\begin{equation}
\mbox{Var}(X^{(m)}) \sim a_1 m^{-1}, \mbox{ as } m \rightarrow \infty
\end{equation}

With {\em slowly decaying variance} we do not get this, instead we observe
\begin{equation}
\mbox{Var}(X^{(m)}) \sim a_2 m^{-\beta}, 0 < \beta < 1 \mbox{ as } m \rightarrow \infty
\label{models:eq3}
\end{equation}

For summation $n{\bar X} = \sum^n_{i = 1}X_i$ we get $u_{n{\bar X}} = n
\lambda t$ and $\sigma^2_{n {\bar X}} = n \lambda t$.  This result
states for the aggregated random variables both mean and variance
increase linearly under summation.  For processes that exhibit slowly
decaying variance the result is that their variance grows faster than
linearly.  For a full example see \S~\ref{appendix:example1}.

Therefore to see the behaviour in equation~\ref{models:eq3} one or
both of the above conditions must fail to hold.  For any renewal
process the random variables will always be identically distributed
leaving either the distribution to have infinite variance and/or there
to be some dependence between the variables.

\section{Fixed interval counting}

Let
\[ T = \{T(k) : k \geq 0 \; k \in {\mathbb N} \} \]
where $T(k)$ is the time of the $k^{th}$ event, conditioned on
\[ T(0) = 0 \]
Let
\[ \begin{array}{c}
R_k = T(k) - T(k-1) \\
R_k > 0 \; \;  k \geq 1 \\
\end{array}
\]
Then this is a renewal process if and only if there exists a
distribution function $F_R(x)$ such that

\[ P(R_k \leq x) = P(R \leq x) = F_R(x) \; \forall x > 0 \mbox{ and } \forall k\geq 1 \]
and the $R_k$ are mutually independent.  That is, inter-event times
are independent and identically distributed.

Let
\[{\mathcal I} = \{i_k : i_k = [ks,(k+1)s) , k \in {\mathbb N} \cup \{0\}\} \]
be a set of non overlapping intervals of size $s$ such that $\cup
{\mathcal I} = {\mathbb R}^+$.

Let
\[{\mathcal C} = \{C_k : C_k = \{ j : T(j) \in i_k \} \} \]
so that
\[
c_k = |C_k|, k \in \{ 0,1,2,\ldots \}
\]
be the number of events in the interval $i_k$, that is $(ks, (k+1)s]$.

Then
\[
P(c_i = k, c_{i-1} = j) = P(c_i = k) \cdot P(c_{i-1} = j), i > 0
\]
is true $\forall s$ if and only if $T$ is a Poisson process.  That is for
general renewal processes $c_i, c_{i-1}$ are not independent.

For example, if we have a renewal process with inter-renewal events
distributed uniformly with mean 0.75, and a counting interval of
length 1.0 then for any interval $c_k$ where the previous interval
$c_{k-1}$ had no events must have at least 1 event in it.

This is because a uniform inter-renewal distribution with mean 0.75
has a maximum inter-renewal time of 1.5.  Since the previous interval
did not have any events ($c_{k-1} = 0$) in it then 1.0 of the maximum
of 1.5 has already occurred.  This means that event must occur within
the current interval ($c_k > 0$).  Hence the number of events that can
occur in the current interval is dependent on the number that occurred
in the previous interval, so the $c_{k-1}$ and $c_k$ are no
independent.

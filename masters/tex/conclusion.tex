\chapter{Conclusion}
\label{conclusion}

\section{What was achieved}

\subsection{Modest resources suffice}

The results produced in the thesis were done using very limited
hardware.  This shows that most people with modest hardware can
reproduce these results for their own networks, widening the
applicability and usefulness of research in this area.

The software used to produce the results is also of modest
proportions.  Apart from the graph plotting, which was done using
GNUplot, all the analysis software was written by myself and is
compilable on any standard C++ compiler under Unix.  This avoids the
need for specialised statistical packages and helps with making the
tools portable to other operating systems and platforms.

While ``super computer'' processing power is not needed, the
calculating of the slowly decaying variance and autocorrelation plots
is computationally intensive and a minimum of a 80486 processor
running at 66 MHz is recommended.

A large amount of storage space is also important, especially if you
want to keep the results even for a short time.  While it is possible
to reproduce all the results from the trace file it is a tedious
process.  Some of the simulations, most notably the superimposed
processes, also produce very large (tens of megabytes) trace files.

\subsection{Samples available for future reference}

A set of traces were taken from around the University campus.  This
provided us with a snapshot of the network in 1994, and allowed for
comparisons between segments and protocols.

This work could usefully be repeated in -- say -- 2 years, so as to
observe whether the traffic behaviour changes as the network matures
and traffic levels increase.

\subsection{The Bellcore results are reproducible}

By reproducing the work done at Bellcore \cite{Bell:1} this thesis has
strengthened both pieces of work and given strong support to the
conclusions they made.  The existence of fractal (self-similar)
behaviour within Ethernet traffic has strong evidence to support it.
This should cause a reassessment of many of the classic models currently
used for congestion prediction in data networks.

Both the mathematics in chapter \ref{models} and the simulations in
chapter \ref{simulation} tie the theoretical models to the observed
results.  The evidence clearly supports the notion that heavy-tailed
renewal processes (renewal processes that have a heavy-tailed
inter-renewal distribution) and superimposed processes produce
self-similar (fractal) behaviour.

\subsection{Producing self-similar behaviour through simulation is not difficult}

The work in chapter \ref{simulation} shows that it is possible to
produce packet traces with self-similar behaviour without difficult
or large amounts of computer processing power.

This gives us a tool to empirically examine self-similar traffic from
various mathematical models.  It also enables people to gain an
understanding of how heavy-tailed renewal processes behave (and
insight into infinite moment distributions).  Such simulation could be
used as an effective learning tool.

\section{What was not achieved}

\subsection{Bugs in the tracing program}

There were several bugs in the tracing program.  They were not removed
because of time restraints and because they did not affect the
results.  As this program will be used in the future to continue to
collect network traffic data around the University these bugs need to
be removed.

The tracing program is also very rudimentary in its user interface.
If this program is to be distributed beyond the experimental stage and
into public use then documentation is required and the user interface
should be improved.

\subsection{Unknown physical processes}

While a strong theoretical model for the observed results now exists
there is still little knowledge about how or why the self-similar
behaviour occurs.  The merging of multiple hosts sending onto the
network is one explanation but without some process corresponding to
a heavy-tailed renewal process it is only a weak conjecture.

This problem was encountered by the Bellcore researchers and they did
not manage to give a conclusive answer.  Without the ability to relate
the theoretical results back to known physical processes it will not
be possible to influence the traffic flow behaviour.  This could be
important when trying to deal with congestion at a source level.

\section{Possible follow up work}

\subsection{Effects of self-similar traffic on network controls}

The existence of self-similar behaviour may require an examination of
current congestion and flow control mechanisms.  As the amount of
network traffic increases and data networks become even more common it
is important that the problems with packet delivery and error
correction using flow controls are addresses.

With the emergence of very busty traffic as normal event on data
networks, which such services as compressed video and music on demand,
congestion control and timely delivery become more and more important.
The current methods of determining buffering within packet switches
and re-transmission ordering may have to examined in light of the
advanced in the theoretical models of self-similar traffic.

\subsection{Looking at network protocol behaviour}

This involves looking at the behaviour bottom up by attempting to
construct accurate models of how network protocols behave.  Work needs
to be done on discovering where the self-similar behaviour originates.
Network protocol behaviour is extremely complicated in real life
because of the complex interaction within operating systems.  A simple
model of a network and protocols which produces self-similar behaviour
would be a valuable tool in understanding what is happening.

Looking at each individual protocol was only done superficially in this
thesis, leaving room for research into how the traffic of each protocol
differs and the possible reasons behind such differences.

\subsection{Further references}

Below are a collection of references that may be of interest
\cite{HAP}
\cite{SDSC:report}
\cite{Lowen:Fract}
\cite{Lowen:1}
\cite{Lowen:2}.
On important reference which was only recently become available (as of
February 1995 in New Zealand) is \cite{taqqu}.  One of its authors
also co-wrote some of the Bellcore papers \cite{Bell:2} \cite{Bell:3}
\cite{Bell:4}.

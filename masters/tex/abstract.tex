\begin{abstract}

The original intention of this thesis was to examine network traffic
entering and leaving the university campus.  The aim was to try to
obtain a measure of how saturated the link was.

While it is simple to give a figure for the average utilisation over
time this is not adequate.  Clearly if the link is heavily utilised,
say over 60\%, then users will experience delays when sending or
receiving data.  The problem is that even on lightly utilised links
delays through congestion can also occur.  This occurs because data
communications traffic levels are not constant but fluctuate over
time.

These fluctuations can occur over very short periods of time giving
rise to the concept of a {\em burst} of traffic.  These bursts of
traffic can be of intensity more than five times that the average
utilisation so that if a user is trying to send data and it coincides
with a burst the user will experience delays.  Traffic which
exhibits these wild fluctuations is known as {\em bursty} traffic.

To this end it is important to gain an insight into the behaviour of
this bursty traffic and try to measure its effect on overall network
performance.

A common assumption in modelling computer networks is that arrivals
occur as a \emph{Poisson process}.  In the thesis we challenge this
assumption and examine the results of doing so.

To this end we decided to take experimental measurements of real
computer networks to try and fit a model to them.  The aim was to
produce a theoretical model which was consistent with real traffic
behaviour.

A collection of mathematical models were investigated.  This included
simulating their behaviour on computers and examining their output in
comparison to that of observed network.  The observed traffic show
\emph{self-similar} (or \emph{fractal}) behaviour.  The thesis
examined which of the investigated models produced similar behaviour.
The final results are summarised in the conclusion with suggestions
for possible areas of further research.

\end{abstract}

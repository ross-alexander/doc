\chapter{Overview of the thesis}

The thesis is divided into five chapters, with an appendix and
bibliography at the end.  Below is a summary of each chapter.

\begin{description}
\item[Chapter \ref{network}]
This chapter introduces computer networks and the software which
controls them.  It discusses in general terms basic computer network
technology and then in more detail those technologies which were used
for the thesis.  The computer networks around the University of
Auckland are also commented on.

\item[Chapter \ref{trace}]
This chapter involves a detailed discussion on how the raw samples
were collected and the problems associated with that exercise.

\item[Chapter \ref{models}]
This chapter is a detailed discussion of the statistical models
investigated in the thesis.  It contains all the underlying
mathematics used later during the simulations.  It also contains an
introduction to the mathematics behind the observed \emph{fractal}
behaviour.

\item[Chapter \ref{results}]
This chapter summaries the results from the observed data primarily in
the form of graphs, with commentary.

\item[Chapter \ref{simulation}]
Here the results of the simulations are discussed, along with how they
were produced and how they compare with the observed data.

\item[Chapter \ref{conclusion}]
The conclusion summarises the work done in the thesis and mentions
possible extensions.  It also discusses some of the problems which
were found during the writing of the thesis.

\item[Appendix \ref{maths}]
This appendix contains mathematical definitions for basic statistical
concepts.  This is background reference for those with limited
statistical knowledge.

\item[Appendix \ref{code}]
Source code listing of software programs written and used for the thesis.

\end{description}

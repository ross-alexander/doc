\documentstyle[epsf]{article}

\title{Determining Validity of Ethernet Sampling Program}
\author{Ross Alexander}
\date{\today}

\begin{document}

\maketitle

\section{Introduction}

Inorder to examine the behaviour of traffic on Ethernet segments a
program was required accurately record all packets transmitted on that
segment.  This was done using an IBM Personal Computer (PC) or
compatible and a program written in C to time stamp and save the first
$n$ octets of each packet.  The captured headers then had to be
written out to permanent storage.

Having written this program a series of controlled experiments were
performed to check the validity and usefulness of the program.  Below
are some of the requirements of interest.

\begin{itemize}
\item	A what rate of packet arrival does packet loss occur.

\item	Does the size of the arriving packets effect packet loss.

\item	In what form does packet loss happen.

\item	How packet loss is effected by writing to permanent storage.
\end{itemize}

\section{The experiment}

\subsection{Design}

This experiment was performed using two machines connected via an
isolated segment of Ethernet.  On one machine a packet generator
executes while the other records all packets seen on the segment and
writes then out to permanant storage.

One important note is the relative processing power of each machine.
While boths machines have compatible architectures one is at least
twice as fast as the other.  This asymmetry enables the recording
program to be ``tested to destruction'' while the packet generating
remains stable.

The experiment was carried out by subjecting the recording machine to
various traffic loads (as specified by the generator's input
parameters) and examining the resulting output file.

\subsection{The packet generator}

The packet generator generates $c$ packets of fixed size $s$ octets.
Between each packet is a delay $d$.  The delay is just a count for a
loop which performs a simple calculation.  This means the delay
depends of the machine it is executing on.  Provided the delay is
linear then this is of no real concern.  These three parameters can be
specified on the command line.

\begin{figure}
\leavevmode
\epsffile{tosh2ipx/size-rate-fit.eps}
\caption{Packet size vs Output rate}
\label{Expr:fig1}
\end{figure}

To examine the behaviour of the packet generator a series of packet
generations was carried out from the slower to the faster machine.
The experiment looked at the maximum rate (in packets per second) the
generator could produce given a specific packet size.  The result can
be seen in Figure~\ref{Expr:fig1}.

The results suggest that beyond packet sizes of 400 octets the
generator's output is bound by the I/O (Input / Output) of the
machine, at about 710,000 octets/second or 693 K per second.  Packet
sizes smaller than this are bound by the processing time needed to
send each packet.

\end{document}

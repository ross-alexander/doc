\chapter{Stochastic Models of Packet Arrivals}

\section{Poisson Process}

\subsection{Definition}

A Poisson process is a counting process
\[ N = \{ N(t) : t \geq 0 \} \]
taking values from
\[ S = \{ 0,1,2,3, \ldots \} \]
conditioned on
\[N(0) = 0\]
and
\[ \mbox{if } s < t \mbox{ then } N(s) \leq N(t) \; \forall s,t \in {\Bbb R} \]
Defined by
\[
\begin{array}{c}
P(N(t+h) = n + m | N(t) = n) = \\
P(N(t+h) - N(t) = m) = \\
\left\{
\begin{array}{rl}
1 - \lambda h + o(h) & m = 0 \\
\lambda h + o(h) & m = 1 \\
o(h) & m > 1 \\
\end{array}
\right. \\
\end{array}
\]

\subsection{Remarks}

The Poisson Process is one of the most common and simple processes in
statistics and operations research.  Poisson processes occur in both
time and space and are characterised by a single parameter $\lambda$.
This parameter $\lambda$ is a measure of the rate or intensity of the
process.

Poisson processes in time occur along a real valued time line,
normally ${\Bbb R}^+$.  They create single events along the line as they
advance towards $+\infty$.

The above definition has the following consequences.

\begin{itemize}
\item	No event may occur at exactly time zero.

\item	The number of events in two non-overlapping time periods are
independent. For example knowing the number of events in the period
$(0, t]$ will not give you any extra information about the number of
events in the period $(t, 2t]$, that is $N(t+s) - N(t) \sim N(s) \;
\forall s,t \in {\Bbb R}^+$.

\item	The number of events in the period $(0, t]$ is distributed
as a Poisson$(\lambda t)$ random varible, that is $P(N(t) = x) = 
\frac{e^{-\lambda t}{\lambda t}^x}{x!}$.
\end{itemize}

It can easily be shown that these give us the following.

\begin{itemize}
\item	The time between any two successive events is distributed
$\mbox{Exponential}(\lambda)$, that is $PT_i - T_{i-1} \geq t) =
e^{-\lambda t}$.

\end{itemize}

\section{Renewal Processes}

A renewal process can be seen as a generalisation of a Poisson
process.  It is important to note that the distribution of the times
betwen renewal events are indenpendent and identically distributed,
for example the Poisson process is a renewal process because of the
inter-arrival time is an indenpendent and identically distributed
exponential distribution.  Renewal processes periodically restart
themselves at the renewal points, forgetting their previous history.
The behaviour between renewal events may be complex.  It also may be
independent of the renewal behaviour.

Often the renewal event is a marked event within a larger process.  An
example is the simple randow walk where the return to the origin or
return to zero can be used.  This is to maintain consistancy with the
concept when a renewal occurs it is as if the process was starting
from time zero.

\subsection{Simple Renewal Processes}

\subsection{Definition}

A renewal process is defined by
\[ T = \{T(k) : k \geq 0 \; k \in {\Bbb N} \} \]
where $T(k)$ is the time of the $k^{th}$ event, conditioned on
\[ T(0) = 0 \]
Let
\[ \begin{array}{c}
R_k = T(k) - T(k-1) \\
R_k > 0 \; \; \forall k \\
\end{array}
\]
and there exists a distribution function $F_R(x)$ such that
\[ P(R_k \leq x) = P(R \leq x) = F_R(x) \; \forall x \mbox{ and } \forall k \]

\subsection{Remarks}

In simple renewal processes the process restarts every time an event
occurs.  This is the simpleist type of renewal process and is very
similar to the Poisson process.  The main difference is that the
Poisson process relies on events having an exponentially distributed
interarrival time.  In a renewal process any probability distribution
may be used for the time until the next renewal.

Because the time until the next renewal has to be non negative
many common distributions such as the normal or Cauchy have to be
modefied.  This can be done by folding the negative half of the
density onto the positive real line.  The simpleist method is sampling
from the distribution and then taking the absolute value.

\subsection{Processes with Internal Structure}

Many processes are more complex then single events.  Between renewals
the process may produce other events.  A birth and death process is a
simple counting process where events arrive and depart from the system
with independent, identically distributed exponential distributions.
We can make the renewal event the event of the system becoming empty.
The events of an empty system becomes our renewal process with the
birth and death process becoming hidden.

\subsection{Processes with Infinite Moments}

One important area of interest are processes which have inter-renewal
times with infinite moments.  We are mainly interested in those
distributions with infinite variance such as Cauchy, $\mbox{T}_2$ and
Pareto.

The research into infinite veriance distribution renewal processes has
lead to the idea of {\em fractal}, or {\em self similar} renewal
processes.  The basis of these names is to do with similar behaviour
over different time spans.  Poisson processes become smoother as you
observe their average behaviour over longer time intervals.  Pure
fractal processes do not smooth so that they are invariant with
respect to the time period over which you observe them.

Such behaviour has serious consequences when applied to traffic models
because it indicates that you will always get bursts of traffic at
every time scale and you can never allocate enough queue resources to
cope with every situation.

\section{Modulated Processes}

Modulated processes are multistate stochastic processes.  They consist
of individual, mutually independent states and a set of stochastic
transitions which govern the process' change from one state to
another.

The process is in exact one of the states at any given time.  While it
in is that state acts in accordance to the behaviour that state
specifies.  The process may change state at any time and must enter
another state straight away.  The rules governing how long the process
stays in a state and which states can follow it may be state dependent
or be controlled by an underlying global process.

\subsection{Markov Modulated Poisson Processes}

\subsubsection{Definition}

Let

\[ {\cal S} = \{s : s = 0,1,2,3,\ldots, N \} \]

be a set of states with associated arrival rates $\lambda_s$, where
$\lambda_s$ is the arrival rate of underlying Poisson process while in
state $s$.  Let

\[ N = \{ N(t) : t \geq 0 \} \]

be the underlying Poisson process of rate $\lambda_{S(t)}$.  Let

\[
S = \{ S(t) : S(t) = s \in {\cal S} \; \; \forall t \geq 0 \; \; t \in {\Bbb R} \}
\]

be the state of the process at time $t$.  Let

\[
P(S(t+h) = k, S(t + \delta) = k \;\;0 \leq \delta < h | S(t) = k) =
e^{\mu_k h})
\]
where $\mu_k \in {\Bbb R}^+$ are state lifetime parameters, with
\[
P(S(t+h) = k | S(t) = j, h = \mbox{inf}\{S(t + \delta) \neq j \; \delta >
0\}) = M[j,k] \; \forall j,k \in {\cal S}
\]
where
\[
M = M[m,n], \; m = n = |{\cal S}|
\]
is a stochastic matrix such that
\[
\sum^n M[m,n] = 1 \; \; \forall m \in {\cal S}
\]

\subsubsection{Remarks}

Markov modulated Poisson processes are the simpleist of the modulated
processes.  Each state defines the arrival rate of an underlying
Markov process.

The amount of time the process stays in a state is exponentially
distributed with a stochastic transition matrix governing the next
state entered.

Markov modulated Poisson processes are often used to generate the
arrivals for standard queuing models.  This is so that more
variability is introduced into the model and to add extra realism
rather than having a constant arrival rate.

\subsection{Generalised Modulated Processes}

\subsubsection{Definition}

Let

\[ {\cal S} = \{s : s = 0,1,2,3,\ldots, N \} \]

be a set of states with associated distribution functions

\[ L_s : (0, +\infty) \rightarrow [0,1] \; s \in {\cal S} \]

Let

\[
S = \{ S(t) : S(t) = s \in {\cal S} \; \; \forall t \geq 0 \; \; t \in {\Bbb R} \}
\]

be the state of the process at time $t$.  Let

\[
P(S(t+h) = k | S(t) = k, S(t + \delta) = k \; 0 \leq \delta < k) =
L_{S(t)}(\delta)
\]
with
\[
P(S(t+h) = k | S(t) = j, h = \mbox{inf}\{S(t + \delta) = k \; \delta >
0\}) = M[j,k] \; \forall j,k \in {\cal S}
\]
given
\[
M = M[m,n] \; \; m = n = |{\cal S}|
\]
be a stochastic matrix such that
\[
\sum^n M[m,n] = 1 \; \; \forall m \; \; m,n \in {\cal S}
\]


\subsubsection{Remarks}

A generalised modulated process consists of a set of states and a
stochastic transition matrix but unlike the Markov modulated Poisson
process it is not limited to only using the exponential distribution
as a lifetime distribution.

Each state has two distributions associated with it.  One is a renewal
process which generates events and the other is a lifetime
distribution.  When the process enters the state a lifetime is
generated from the lifetime distribution.  The process will remain in
that state for the period of the lifetime.

While in that state a renewal process will run generating events.  If
an event is generated with would have occured beyond the lifetime of
the state then it is ignored.  Any generalised renewal process is able
to run provided it is independent of all other states.

Once the lifetime of state has expired then the process choses a new
state to enter based on the probililities in the transition matrix.
The transition matrix is constant and independent of any activity
within any of the states.

\section{Aggregation Processes}

\subsection{Definition}

Let
\[ {\cal C} = \{c:c = 0,1,2,3,\ldots, N\} \]
be a set with associated general renewal processes
\[R_c(t), \; c \in {\cal C} \]

Let
\[ T_c = \left< T_{c_1}, T_{c_2}, \ldots \right> \]
 be the sequence of renewal times of the process $R_c$.

Let 
\[ T = \left<T_1, T_2, T_3, \ldots \right> = \bigcup^{\cal C}T_c \]
 be an ordered sequence such that 
\[ T_i \leq T_j \; \forall i,k \in {\Bbb N} \]

\subsection{Remarks}

With an aggregated process multiply renewal or modulated processes run
concurrently with their result event traces merged together.  In most
models each process is identical and all must be mutually independent.

The idea behind aggregated processes is to capture complex behaviour
using simply renewal processes with few parameters.  This would give
us a much simplier model without loosing the complex nature of the
real behaviour.


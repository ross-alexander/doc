\chapter{Network Tracing}

A network (or packet) trace is a record of activity of a physical
network over a period of time.  A network trace can contain
information about the length of each packet, the time it was seen and
part or all of the actual packet contents.

The emphasis of this dissertation is traffic intensity with respect to
time.  For this packet inter-arrival times are needed.  This is done
by recording a timestamp in the trace.  The more accurate the time
stamp the finer the analysis can be performed.

For packet size analysis the length of the packet has to be recorded.
Because packet sizes are a function of higher level protocols for any
in depth analysis the origin of the packet has to be obtained.  This
can be done by examining the contents of each packet and extracting
the relevant information.  This is a time consuming task so it cannot
be done during the execution of the tracing program.  Instead the
first forty to fifty bytes of each packet is recorded for later
processing.

Packet traces are normally stored as large binary files.  Programs can
then examine these files and perform statistical analysis on the data.
Recording a trace over any reasonable length of time produces large
files, in the order of megabytes or tens of megabytes.

\section{Producing a packet trace}

The packet traces of Ethernet segment were made using an IBM PC
directly connected to the ethernet.  Because of the broadcast nature
of ethernet it is easy to record all packets sent on the network.  The
program records a time stamp, the packet size and the first $n$ bytes
of the packet, where $n$ is defined at compile time.

The number of packets the program records is set at compile time, and
normally ranges from ten thousand to half a million.  In a later version
of the program it should be possible to specify the number of minutes
the program is to run.

\subsection{Format of the trace file}

\begin{figure}
\leavevmode
\epsffile{trace-format.eps}
\caption{Format of the trace file}
\label{trace:format}
\end{figure}

The trace file has an eight byte header containing four records, each
a two byte big endian word.  The contain the length of the time stamp
(in octets), the length of the packet size, the number of octets of
packet header and unused respectively (see Figure~\ref{trace:format}).

Following the header are the actual trace records, which contain the
time stamp, the size of the packet (this includes the ethernet header
but not the frame sequence check) and the first $n$ of the packet.

\subsection{Hardware unused to produce the traces}

The hardware used was a standard IBM PC compatible with a Ethernet
controller.  This was used because it was easy to write the software
needed to produce the trace and the hardware is widely, and cheaply,
available.  The IBM PC architecture does have several drawbacks.

\begin{itemize}
\item	The clock used to generate the time stamps is limited in
accuracy to approximately 1 ms.

\item	The input/output bus is limited to transfers of about 700
kilobytes per second.  Very long bursts could be a problem.

\item	Interrupt events block other interrupts, causing problems
with the timer.
\end{itemize}

The Ethernet controller has an eight kilobyte buffer so as the reduce
actual packet loses.  This may effect the analysis as queued packets
will be seen by the program as having a fixed inter arrival time.

\subsection{Software used to generate the trace file}

The software is written in C and then compiled.  It is based on {\em
packet driver} software.  Packet drivers are pieces of code which
isolate the programmer from the specific hardware of each controller,
presenting a standard interface regardless of the actual controller.
The enables programmers to write much simpler programs at the expense
of a small extra overhead.

When a packet arrives at the ethernet controller it generates a signal
to interrupt the processor and process the packet.  The packet driver,
while in interrupt mode (that is, the processor has suspended what is
was currently doing and executes a special routine called the
interrupt handler) then calls a user program with the size of the
packet.  This then returns a pointer into memory for the contents of
the packet to be copied (this includes the packet header).  The driver
then copies the packet and calls a second user program routine to tell
it that the copy has been completed.  Once second routine has exited
the packet driver then returns from interrupt mode, and the processor
restarts what is was originally doing.

The problem with interrupt mode is that code executing while in
interrupt mode cannot itself be interrupted (this is not strictly
correct but is true for packet drivers).  Interrupts which occur while
this happens are placed in a priority queue.  For this reason
interrupt handlers must do as little work as possible.  In general
programs just set flags and update simple data structures during
interrupt mode.

The program itself loops repeatedly checking these flags for change.
If a flag is set it then processes the packet from the head of the
data structure.  This looping is called {\em polling}.

\subsubsection{Data Structures Used}

\begin{figure}
\leavevmode
\epsffile{trace-structures.eps}
\caption{Data Structures Used in Tracer}
\label{trace:structures}
\end{figure}

The data structure consists of two fixed fixed buffers
(Figure~\ref{trace:structures}).  While one is being used the other is
either waiting to be written out to a file or is empty.  Each buffer
has a fixed number of records into which packet headers can be
written.

The {\em Output File} is a standard C file pointer and is defined as
part of the C {\em stdio} (standard input/output) library.  This
enables the program to be portable to different operating systems.

There are also variables such as {\em Initial Time} and {\em Total
Packet Count} for determining when the program should terminate.

\subsubsection{Flow of polling loop}

\begin{figure}
\leavevmode
\epsffile{trace-flow.eps}
\caption{Flow of polling loop}
\label{trace:flow}
\end{figure}

The program has three stages, the first initialises the required
variables and data structures, including the timer and packet driver.
The second is a repeated loop which continually checks to see if a
buffer needs to be written out to a file and the last is the final
clean up before exiting (Figure~\ref{trace:flow}).

There are various reasons for the program to terminate.  In the normal
course of events either the number of packets recorded reaches a set
amount or a running of the program exceeds a certain time limit.  Both
these values can be specified when the program is loaded into memory.
If the writing to the file should fail for some reason then the
program will also terminate.  Generally this will be because the disk
is full but other errors can occur.

\subsubsection{Flow during interrupt}

\begin{figure}
\leavevmode
\epsffile{rcvPkt1-flow.eps}
\caption{Flow for first packet driver call}
\label{trace:rcvpkt1}
\end{figure}

\begin{figure}
\leavevmode
\epsffile{rcvPkt2-flow.eps}
\caption{Flow for second packet driver call}
\label{trace:rcvpkt2}
\end{figure}

When an interrupt is received by the packet driver it in turn calls
the user with the size of the packet and expects a pointer into memory
as a return result.  It also stores the current timer value and the
size of the packet into the current record
(Figure~\ref{trace:rcvpkt1}).

The packet driver then copies the contents of the packet into the
specified memory and calls a second user routine.  This simply updates
various counters and exits without any return value
(Figure~\ref{trace:rcvpkt2}).

\subsection{Known Problems}

\subsubsection{Errors with the timer}
\label{trace:timerprob}

The timer works by having an internal 16 bit counter, say $c$,
starting at 0xFFFF (65535), and counting down to 0x0000, and then it
resets itself automatically and starts counting down again.  When it
reaches 0x0002 it causes an interrupt and a counter $t$ kept by the
timer code is increased.  The counter is split into upper and lower
bytes, $c_u$ and $c_l$ respectively.  The value returned from the
timer is
\[
2^{8}t + (256 - c_u)
\]
which should theoretically always give us the correct result.  But
because the timer is asynchronous to the main processor so if $t$
should fail to be increased at the correct time a may problem occur.
With the Intel 80x86 processor a flag allows or blocks interrupts from
occurring.  If this flag is set then interrupts get queued, waiting
until the flag is cleared so they can occur.

If a packet arrives when $c$ is close to zero then an interrupt occurs
and the packet driver is called.  The packet driver sets the interrupt
flag (since another packet may be received while the driver is still
processing the first and if it was allowed to interrupt again the code
would probably crash).  While the driver is processing the interrupt
the timer reaches 0x0002, causes and interrupt, which is queued,
counts a little more and resets itself.

The tracing code then reads the value from the timer code, which is
now incorrect, since $t$ has not yet been incremented, and stores it
in within the current buffer.  Afterwards the packet driver clears the
interrupt flags and exits, the timer interrupt occurs, $t$ is
incremented to the correct value, and the timer value is what it
should be.

The problem occurs because of a limitation of the Intel 80x86
architecture and cannot be avoided without considerable modification
of the packet driver code to stop it from having the interrupt flag
set while calling the user program code.  It is unclear whether this
is even feasible.

Because this problem is known it is correctable in later processing
without considerable difficultly and so rather than try to fix it at
its origin it is simpler to make the corrections.

\subsubsection{Machine crashes after program exits}

After running the program it is necessary to reboot the machine it ran
on.  This is more an inconvenience rather than a problem.  The reason
for the problem is unknown but the timer code is the most suspect.
The amount of time needed to fix this problem is not worth the it and
since it does not affect the results it can safely be ignored.

\section{Processing trace files}

Once the trace file has been recording it is transfered to a Unix host
where it is processed.  The first processing done is the correcting
the timer problem (Section~\ref{trace:timerprob}).  The file may then
be divided up, based on the various protocols embedded in the packet,
and is then used to generate packet frequency (by time interval)
data.

\subsection{Correcting the timer problem}

\subsection{Producing frequency data}

The process of producing discrete time interval packet and octet
counts is very simple.  The program {\em interval} is given the slot
size in milliseconds and whether the output should be a binary file or
and ASCII text file.

\begin{figure}
\begin{verbatim}
     4649    65   11524
     4650    66   12652
     4651    61   12261
     4652    56   10319
     4653    54   12231
     4654    57   10219
     4655    58   10704
     4656    66   12109
     4657    49   10574
     4658    60   11231
     4659    49   10424
\end{verbatim}
\caption{Sample output of {\em interval}}
\label{interval:figure1}
\end{figure}

The text output of {\em interval} can be seen in
Figure~\ref{interval:figure1}.  The first column is the interval
number, the second the packet count and the third the total number of
octets.  The binary output has the same format except the entries are
stored as 32 it integer values.  The binary file also has the total
number of rows and columns stored at the end of the file as 32 bit
integer values.  This is to help programs which read the file in.

\subsection{Dividing trace into protocols}

To provide information in individual protocols packet traces need to
separated.  This is done by a program {\em demux} (for demultiplex),
which takes each packet header in turn, analyses it and decides via a
set of rules which output file to store it in.

Currently the rules provide for IP, AppleTalk, IPX and most of DECnet.
As IP, IPX and AppleTalk make up the balk of all the University's
network traffic this is adequate.

\section{Testing the packet tracer}

\subsection{Introduction}

Inorder to examine the behaviour of traffic on Ethernet segments a
program was required accurately record all packets transmitted on that
segment.  This was done using an IBM Personal Computer (PC) or
compatible and a program written in C to time stamp and save the first
$n$ octets of each packet.  The captured headers then had to be
written out to permanent storage.

Having written this program a series of controlled experiments were
performed to check the validity and usefulness of the program.  Below
are some of the requirements of interest.

\begin{itemize}
\item	A what rate of packet arrival does packet loss occur.

\item	Does the size of the arriving packets effect packet loss.

\item	In what form does packet loss happen.

\item	How packet loss is effected by writing to permanent storage.
\end{itemize}

\subsection{The experiment}

\subsubsection{Design}

This experiment was performed using two machines connected via an
isolated segment of Ethernet.  On one machine a packet generator
executes while the other records all packets seen on the segment and
writes then out to permanent storage.

One important note is the relative processing power of each machine.
While both machines have compatible architectures one is at least
twice as fast as the other.  This asymmetry enables the recording
program to be ``tested to destruction'' while the packet generating
remains stable.

The experiment was carried out by subjecting the recording machine to
various traffic loads (as specified by the generator's input
parameters) and examining the resulting output file.

\subsubsection{The packet generator}

The packet generator generates $c$ packets of fixed size $s$ octets.
Between each packet is a delay $d$.  The delay is just a count for a
loop which performs a simple calculation.  This means the delay
depends of the machine it is executing on.  Provided the delay is
linear then this is of no real concern.  These three parameters can be
specified on the command line.

\begin{figure}
\leavevmode
\epsffile{size-rate-fit.eps}
\caption{Packet size vs Output rate}
\label{Expr:fig1}
\end{figure}

To examine the behaviour of the packet generator a series of packet
generations was carried out from the slower to the faster machine.
The experiment looked at the maximum rate (in packets per second) the
generator could produce given a specific packet size.  The result can
be seen in Figure~\ref{Expr:fig1}.

The results suggest that beyond packet sizes of 400 octets the
generator's output is bound by the I/O (Input / Output) of the
machine, at about 710,000 octets/second or 693 K per second.  Packet
sizes smaller than this are bound by the processing time needed to
send each packet.

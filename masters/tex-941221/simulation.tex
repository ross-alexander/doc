\chapter{Simulation}

\section{Introduction}

\section{Markov Process Simulation}

\subsection{The Model}

A {\em Continuous time Markov Process} is a collection of states $S =
{1,2,3,\ldots}$, and a transition matrix J.  The process is always in
exact one state $s = s(t)$ at any one time.  When the process leaves
state $s$ it jumps to another state with probabilities given by the
$s^{th}$ row of $J$.

After the process enters a state it will remain in that state for a
given period of time.  This time is calculated from some probability
distribution $L_s$.

While the process is in a given state that state can generate events.
The distribution of the inter-arrival time between each event is
calculated from some probability distribution $I_s$.

Three different probability distributions are used in the simulations.
These are the {\em exponential} distribution, the {\em scaled uniform}
and a {\em deterministic distribution}.

\subsection{The Program}

The program reads in a parameter file and produces a trace file
(Figure~\ref{trace:format}).

\begin{verbatim}
  simulate [-t time in seconds] parameter files
\end{verbatim}

The format of the parameter 

\section{Simulation of simple poisson process}

Using the simplist model available, that is exponentially distributed
inter-arrival times, we can produce samples traces to examine.  This
simulation is not meant to be comparable to the real traces, but
rather a counter example to show that such a model does not fit
Ethernet traffic patterns.

It is also to show that the simulation techniques work and the tools
developed to produce the results are robust.

The simple simulations are done using a continuous time Markov model.
With this model a process is a state $s$ for some time $t$, derived
from a distribution $T_s$.  While in state $s$ it produces events from
a distribution $R_s$ with mean $\mu_s$.  After $t$ time intervals have
passed the process leaves state $s$.  It then enters a new state
$s_{new}$, with the probabilities of switching from state $s$ to
states $s_{new}$ given as a probability matrix $J$.

\subsection{Single State with Exponential Arrival Distribution}

\begin{figure}
\leavevmode
\epsfysize=3in
\epsffile{sim-1-1s.eps}
\caption{Poisson Process Simulation with Exponential Distribution}
\label{simple:sim.1.eps}
\end{figure}

The first simulation {\em sim-1} is a Poisson Process with rate
$\lambda = 0.05$ packets per millisecond.  This results in an
exponentially distributed inter-arrival time with mean $\mu = 20$.  A
sample trace can be seen in Figure~\ref{simple:sim.1.eps}.

\subsection{Single State with Uniform Arrival Distribution}

\begin{figure}
\leavevmode
\epsfysize=3in
\epsffile{sim-2-1s.eps}
\caption{Poisson Process Simulation with Uniform Distribution}
\label{simple:sim.2.eps}
\end{figure}

\section{Markov Modelated Poisson Process}

\subsubsection{An Example}

For an example I have choosen a three state process.

\begin{tabular}{||l||l|l||l|l||} \hline
State & \multicolumn{2}{||l||}{Event Distribution} &
\multicolumn{2}{||l||}{Lifetime Distribution} \\
 & Distribution & Mean & Distribution & Mean \\ \hline \hline
0 & Exponential & 10 & Exponential & 100 \\ \hline
1 & Deterministic & 2 & Uniform & 40 \\ \hline
2 & Deterministic & 1 & Deterministic & 15 \\ \hline
\end{tabular}

\section{Aggreation of Multiply Processes}

\subsection{Markov Processes}

\subsection{Non-exponential Processes}

\section{Infinite Varience Simulation}

\subsection{Introduction}

\subsection{Pareto Distribution}

\subsection{Cauchy and T Distributions}


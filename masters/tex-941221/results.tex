'\chapter{Experimental Results}

\begin{figure}
{\small \begin{tabular}{|l|l|l|l|l|l|} \hline
Date & Location & Traffic Mix & Load & Time & Packet Count \\ \hline \hline
14 May & Computer Centre & AppleTalk, Telnet & Light & 14 minutes & 400,021 \\
 & Staff Network & Novell client & & & \\ \hline
13 June & Gateway to & Various IP & Light & 102 minutes & 329,379 \\
 & Outside World & & & & \\ \hline 
1 August & Statistics & IP, mainly NFS & Medium & 118 minutes & 400,048 \\
 & & & & & \\ \hline
1 September & Commerce & NetBIOS from & Medium & 94 minutes & 800,002 \\
 & & Windows NT & & & \\ \hline
\end{tabular}}
\caption{Traffic Samples from around The University of Auckland}
\label{trace:sampletable}
\end{figure}

For this section I have taken a four samples from around the
University (See Figure~\ref{trace:sampletable}) over a period of five
months during the year.  These have been chosen to give a wide
selection of different protocols and traffic conditions.

\section{Mean and Variance of Samples}

\subsection{Results by Segment}

\begin{figure}
{\begin{tabular}{|p{4cm}|r|r|r|} \hline
Sample & Interval & Mean & Variance \\ \hline \hline
Gateway Network & 10s & 537.5 & 13,300 \\
		& 1s & 53.7 & 305.390 \\
		& 100ms & 5.37 & 11.483 \\
		& 10ms & 0.537 & 0.744 \\ \hline
Subnet 1 & 10s & 3,008 & 1,974,893 \\
	 & 1s & 300.7 & 35,633 \\
	 & 100ms & 30.08 & 546 \\
	 & 10ms & 3.008 & 8.921 \\ \hline
Subnet 93 & 10s & 561 & 181,669 \\
	  & 1s & 56.1 & 4,211.3 \\
	  & 100ms & 5.61 & 83.439 \\
	  & 10ms & 0.561 & 2.058 \\ \hline
Commerce  & 10s & 1,408.5 & 312,670 \\
	  & 1s  & 140.85  & 4,768.9 \\
	  & 100ms & 14.085 & 81.98 \\
	  & 10ms & 1.408 & 2.896 \\ \hline 
\end{tabular}}
\end{figure}

\section{Frequency/Time Graphs of Results}

Below are a collection of graphs showing packet count against time.
Although little statistical information can be gained from these
graphs it can be seen visually the wide range of different behaviours
exhibited by Ethernet traffic.

\subsection{Graphs by Segment}

An attempt was made to try and get a wide sample of different traffic
behaviours from around the university campus.

\subsubsection{Subnet 1 of the University Network}

\begin{figure}
\leavevmode
\epsfysize=3in
\epsffile{snet1-1s-freq.eps}
\caption{Subnet 1 with time interval 1 second}
\label{simple:snet1.1s.freq}
\end{figure}

Figure~\ref{simple:snet1.1s.freq} shows the 1,000 seconds of samples
taken from subnet 1.  This traffic has a steady background level of
traffic with frequent bursts.

\subsubsection{Subnet 93 of the University Network}

\begin{figure}
\leavevmode
\epsfysize=3in
\epsffile{snet93-1s-freq.eps}
\caption{Subnet 93 with time interval 1 second}
\label{simple:snet93.1s.freq}
\end{figure}

Figure~\ref{simple:snet93.1s.freq} shows a much lightly loaded segment
with minimal background traffic but still exhibiting very bursty
traffic.

\subsubsection{External Gateway Network of the University}

\begin{figure}
\leavevmode
\epsfysize=3in
\epsffile{gatew-1s-freq.eps}
\caption{Gateway network with time interval 1 second}
\label{simple:gatew.1s.freq}
\end{figure}

Figure~\ref{simple:gatew.1s.freq} has an extremely light load with a
considerable amount (with respect to the total) of the traffic as a
steady background flow.  The bursts are much smaller in conparison to
the other segments, with 

\subsection{Graphs by Protocol}


Below are three graphs showing the most prevalent protocols used
around the university, that is IP, AppleTalk and Netware IPX.

\subsubsection{Internet Protocol}

\begin{figure}
\leavevmode
\epsfysize=3in
\epsffile{snet1-ip-1s.eps}
\caption{IP traffic on subnet 1 with time interval 1 second}
\label{simple:snet1.ip.1s.freq}
\end{figure}

Figure~\ref{simple:snet1.ip.1s.freq} shows Internet Protocols traffic.
This graphs shows a reasonable (with respect to the peak maximums)
amount of background traffic.  This is most likely a result of a
steady level of remote login and mail traffic.

\subsubsection{AppleTalk}

\begin{figure}
\leavevmode
\epsfysize=3in
\epsffile{snet1-apt-1s.eps}
\caption{AppleTalk traffic on subnet 1 with time interval 1 second}
\label{simple:snet1.apt.1s.freq}
\end{figure}

Figure~\ref{simple:snet1.apt.1s.freq} shows AppleTalk traffic on
subnet 1.  Because there is very little AppleTalk traffic on subnet 1
events become more pronounced and noticable.  AppleTalk has periodic
broadcasts to relay routing and directory services (zone maps)
information.

\subsubsection{Netware IPX}

\begin{figure}
\leavevmode
\epsfysize=3in
\epsffile{snet1-nov-1s.eps}
\caption{IPX traffic on subnet 1 with time interval 1 second}
\label{simple:snet1.nov.1s.freq}
\end{figure}

Figure~\ref{simple:snet1.nov.1s.freq} show Netware traffic on subnet 1.
While it looks like there is considerable IPX traffic in fact there is
little file transfer traffic.  The large bursts of traffic come from
Netware's service announcements.  Every two minutes Netware capable
routers broadcast all available Netware services, such as file
servers, printing spoolers and mail exchanges.  Due to the number of
these services around the university is amounts to over 500 kilobytes
of information.

\section{Results over short time periods}

\section{Packet Count vs Octet Count}

\section{Slow decay varience}

